\documentclass{article}

\usepackage{natbib}

\usepackage[submitted]{mydraft}
%\usepackage[draft]{mydraft}
\papertarget{the IFIP WC-DSL}

\begin{document}
\title{A Haskell Hosted DSL for\\Writing Transformation Systems}
\author{Andy Gill}

\maketitle
%\enabledraft{Submitted to the IFIP Working Conference on Domain Specific Languages\\\sdf\\sdfsdfsdfsgfsg\\sdsgsgsdg\\sdfsgg}

\begin{abstract}
KURE is a Haskell hosted Domain Specific Language (DSL) for writing transformation systems
based on rewrite strategies.
%1. State the problem 
When writing transformation systems, a significant amount of engineering effort goes into setting up plumbing
to make sure that specific rewrite rules can fire.
%2. Say why it’s an interesting problem 
Systems like Stratego and Strafunski provide most of this plumbing as infrastructure, allowing the DSL user
to focus on the rewrites rules.
%3. Say what your solution achieves 
KURE is a strongly typed strategy control language in the tradition of Stratego and Strafunski.
It is intended for writing reasonably efficient rewrite systems, and
makes use of type families to provide a delimited generic mechanism for tree rewriting,
provides support for efficient identity rewrite detection.
%4. Say what follows from your solution
%KURE is therefore an excellent DSL for implementing rewriting and transformation systems in Haskell.
\end{abstract}


\section{Introduction}

Sometimes its easy to say what you want to do, but tedious to get the scaffolding in place. This is an ideal candidate for a Domain Specific Language (DSL) where the language provides the scaffolding as a pre-packaged service. 
For example, scaffolding is needed to promoting local, syntax directed rewrites into a global context. Take the first case rewriting rule from the Haskell 98 Report~\cite{Haskell98Book} which says:

\[
\begin{array}{@{}cl}
(a)&\verb@case @$e$\verb@ of { @$alts$\verb@ } @$=$\verb@ (\@$v$\verb@ -> case @$v$\verb@ of { @$alts$\verb@ }) @e\\
&${\rm where $v$ is a new variable}$\\
\end{array}
\]

This rule is used to simplify a case over an arbitrary expression, and convert it to a case over a {\em variable\/}.
We can express this rule directly in Haskell, using a rewrite in the Haskell Syntax provided by Template Haskell~\cite{metahaskell}, using the function \verb|rule_a|.

This rule reflects our syntax rewrite rule almost directly,
and cleanly utilizing a monadic name supply for a fresh variable.
The rule also acts locally, and needs additional machinery to perform this rewrite on whole programs. 
Strategic programming is a paradigm which builds on rewrite primitives like \verb|rule_a|
by using combinators to construct complex and powerful rewrites and transformation systems. 

This paper introduces KURE, a strategic programming DSL hosted in Haskell being developed
at the University of Kansas. KURE provides a small set of combinators that can be used
to build parameterized term rewriting and general user-defined rule application. In this paper, we discuss
in detail the design of KURE, and how we used the tradition of type-centric DSL design
to drive our implementation. This paper makes the following specific contributions.
\begin{itemize}        

\item We move both rewrites and rewrite strategies into a strongly typed world.
Throughout the paper we make detailed comparisons with previous attempts to providing typing to 
rewrite strategy systems.

\item More specifically, we provide typing for term traversing strategies
using a new capability, a light\-weight and
customized generic term traversal mechanism implemented using associated type synonyms~\cite{Chakavarty:AssocTypeSyn}.

\item We introduce the ability to record equality over terms, which is traditionally a weakness of purely functional rewrite engines.
Equality is important because the use-case for many transformations in practice is to iterate until nothing else can be achieved.

\item We abstract our combinators explicitly over a user-defined context, allowing concerns like environment or location to be 
represented.
We show the generality of this contribution by implementing path selection.
This design choice allows a particularly clean relationship between the semantics being hosted by the DSL, and the implementation using our DSL.


\item We lay out the methodology used in the design of our hosted DSL, with the intent that others can
use our design template.

%\item KURE is a meta DSL. It in intended to allow people to write rewrite rules for other to use as a DSL. 
%We justify this objective using a case study of an applying KURE to a System-F syntax.

\end{itemize}

In this paper, we distinguish between DSL code written using KURE and the implementation of KURE itself by placing code
fragments that are uses of the KURE DSL (or similar) inside boxes, and code fragments that are describing the KURE
internals without boxes.

KURE is both practical and useful, and builds on many years of research and development into rewrite strategies, especially the work of Stratego~\cite{Vis04.strategoxt} and Strafunski ~\cite{LV02-PADL}, and our own experiences with HERA~\cite{gillhera06}.
We compare KURE to these systems throughout this paper, as well as make different design decisions explicit.
We summarize the differences in section~\ref{sec:related}.
In recognition to the contributions make by these systems, we reuse where possible the naming conventions of Stratego and Strafunski. 

\newpage\section{Introducing Strategic Programming}\label{intro:stra-prog}

Stratego and other strategic programming languages provide 
a mechanism to express rewrites as primitive strategies, and 
a small set of combinators that act on strategies, giving the
ability to build complex custom strategies.
In this section, we introduce the basic combinators in strategic programing used in Stratego.
This will guide our design of KURE.




\end{document}
